%iffalse
\let\negmedspace\undefined
\let\negthickspace\undefined
\documentclass[journal,12pt,twocolumn]{IEEEtran}
\usepackage{cite}
\usepackage{amsmath,amssymb,amsfonts,amsthm}
\usepackage{algorithmic}
\usepackage{graphicx}
\usepackage{textcomp}
\usepackage{xcolor}
\usepackage{txfonts}
\usepackage{listings}
\usepackage{enumitem}
\usepackage{mathtools}
\usepackage{gensymb}
\usepackage{comment}
\usepackage[breaklinks=true]{hyperref}
\usepackage{tkz-euclide} 
\usepackage{listings}
\usepackage{gvv}                                        
%\def\inputGnumericTable{}                                 
\usepackage[latin1]{inputenc}                                
\usepackage{color}                                            
\usepackage{array}                                            
\usepackage{longtable}                                       
\usepackage{calc}                                             
\usepackage{multirow}                                         
\usepackage{hhline}                                           
\usepackage{ifthen}                                           
\usepackage{lscape}
\usepackage{tabularx}
\usepackage{array}
\usepackage{float}


\newtheorem{theorem}{Theorem}[section]
\newtheorem{problem}{Problem}
\newtheorem{proposition}{Proposition}[section]
\newtheorem{lemma}{Lemma}[section]
\newtheorem{corollary}[theorem]{Corollary}
\newtheorem{example}{Example}[section]
\newtheorem{definition}[problem]{Definition}
\newcommand{\BEQA}{\begin{eqnarray}}
\newcommand{\EEQA}{\end{eqnarray}}
\newcommand{\define}{\stackrel{\triangle}{=}}
\theoremstyle{remark}
\newtheorem{rem}{Remark}

% Marks the beginning of the document
\begin{document}
\bibliographystyle{IEEEtran}
\vspace{3cm}

\title{CONIC SECTION}
\author{EE24BTECH11038 - MALAKALA BALA SUBRAHMANYA ARAVIND}
\maketitle
\newpage
\bigskip

\renewcommand{\thefigure}{\theenumi}
\renewcommand{\thetable}{\theenumi}
\section{section b}
\begin{enumerate}
\item[31.]  A hyperbola passes through point  p$(\sqrt2,\sqrt3)$  and  has  foci  at 
      ($\pm2,0)$. Then  the  tangent  to  this  hyperbola at P also passes through the point :
      
      \hfill{(JEE M 2017)} 

    (A)($-\sqrt2,-\sqrt3$) \hspace{2cm} (B)($3\sqrt2,2\sqrt3$)\\


    (C)($2\sqrt2,3\sqrt3$) \hspace{2cm}  (D)($\sqrt3,\sqrt2$)\\

\item[32.]  The radius of a circle, having minimum area, which touches the curve $y=4-x^2$ and the lines ,$y=|{x}|$ is : 

      \hfill{(JEE M 2018)}

     (A)4($\sqrt2$+1) \hspace{2cm} (B)2($\sqrt2$+1)\\

     (C)2($\sqrt2$-1)  \hspace{2cm}  (D)4($\sqrt2$-1)\\

\item[33.] Tangents are drawn to the hyperbola 4$x^2-y^2$=36 at the points P and Q. If  these tangents intersect  at the point T(0,3) then the area (in sq.units) of $\Delta$ PTQ is:
    
     \hfill{(JEE M 2018)}

     (A) 54$\sqrt3$  \hspace{2cm}  (B)60$\sqrt3$\\

     (C) 36$\sqrt3$  \hspace{2cm}  (D)45$\sqrt5$\\

\item[34.] tangent and normal are drawn at P(16,16) on the parabola $y^2=16x$,
which is intersect the axis of the parabola at A and B,respectively.If C is the centre of the circle through the points P,A and B and $\angle$ CPB=$\theta$, then the value of tan$\theta$ is :

     \hfill{(JEE M 2018)}

     (A)2    \hspace{2cm}  (B)3\\

     (C)4/3  \hspace{2cm}(D)1/2\\

\item[35.]  Two sets A and B are as under:\\
A=\{(a,b)$\in$RXR : $|{a-5}|<1$ and $|{b-5}|<1$\};\\
B=\{(a,b)$\in$RXR : 4$(a-6)^2$ + 9$(b-5)^2$$\leq$36\}.Then:

    \hfill{(JEE M 2018)}

    (A) A$\subset$B \\   (B)A$\cap$B

    (c)neither A$\subset$B nor B$\subset$A\\
    (D)B$\subset$A\\

\item[36.] If the tangent at (1,7) to the curve $x^2=y-6$ touches the circle $x^2+y^2+16x+12y+c=0$ then the value of c is :

     \hfill{(JEEM 2018)}

    (A)185\hspace{1cm}(B)85\hspace{1cm}(C)95\hspace{1cm}(D)195\\

\item[37.] Axis of a parabola lies along X-axis.If its vertex and focus are at a distance 2 and 4 respectively from origin, on the positive X-axis then which of the following points does not lie on it? 

     \hfill{(JEE M 2018)}

    (A)(5,2$\sqrt6$)\hspace{2cm}  (B)(8,6)\\

    (C)(6,4$\sqrt2$)\hspace{2cm}   (D)(4,-4)\\

\item[38.] Let 0$<$$\theta$$<$$\pi/2$  .If the eccentricty of the hyperbola $\cfrac{x^2}{cos^2\theta}$ - $\cfrac{y^2}{sin^2\theta}$ = 1 is greater than 2, then the length of its latus rectum lies in the interval:

    \hfill{(JEE M 2019-9 Jan(M)}

    (A)(3,$\infty$) \hspace{2cm} (B) (3/2,3]\\

    (C)(2,3]  \hspace{2cm}   (D)(1,3/2]\\

\item[39.] Equation of a common tangent to the circle $x^2+y^2-6x=0$ and the parabola $y^2=4x$, is:

    \hfill{( JEE M 2019-9 Jan(M))}

   (A)2$\sqrt{3}$y=12x+1 \hspace{2cm} (B)$\sqrt{3}y$=x+3\\

   (C)2$\sqrt{3}y=-x-12$  \hspace{2cm}  (D)$\sqrt{3}$y=3x+1\\

\item[40.] If the line y=mx+7$\sqrt{3}$ is normal to the hyperbola $\cfrac{x^2}{24}$-$\cfrac{y^2}{18}$ then a value of m is: 

   \hfill{(JEEM 2019-9 April(M))}

  (A)$\sqrt{5}/2$ \hspace{2cm}  (B)$\sqrt{15}/2$

  (C)$2/\sqrt5$ \hspace{2cm}    (D)$3/\sqrt{5}$

\item[41.] if one end of a focal chord of the parabola,$y^2=16x$ is at (1,4),then the length of this focal chord is :

      \hfill{( JEE M 2019-9 Jan(M))}

   (A)25\hspace{1cm}(B)22\hspace{1cm}(C)24\hspace{1cm}(D)20

    
    
     
     
  \end{enumerate}
\end{document}
