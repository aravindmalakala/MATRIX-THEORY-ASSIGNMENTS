%iffalse
\let\negmedspace\undefined
\let\negthickspace\undefined
\documentclass[journal,12pt,twocolumn]{IEEEtran}
\usepackage{cite}
\usepackage{amsmath,amssymb,amsfonts,amsthm}
\usepackage{algorithmic}
\usepackage{graphicx}
\usepackage{textcomp}
\usepackage{xcolor}
\usepackage{txfonts}
\usepackage{listings}
\usepackage{enumitem}
\usepackage{mathtools}
\usepackage{gensymb}
\usepackage{comment}
\usepackage[breaklinks=true]{hyperref}
\usepackage{tkz-euclide} 
\usepackage{listings}
\usepackage{gvv}                                        
%\def\inputGnumericTable{}                                 
\usepackage[latin1]{inputenc}                                
\usepackage{color}                                            
\usepackage{array}                                            
\usepackage{longtable}                                       
\usepackage{calc}                                             
\usepackage{multirow}                                         
\usepackage{hhline}                                           
\usepackage{ifthen}                                           
\usepackage{lscape}
\usepackage{tabularx}
\usepackage{array}
\usepackage{float}


\newtheorem{theorem}{Theorem}[section]
\newtheorem{problem}{Problem}
\newtheorem{proposition}{Proposition}[section]
\newtheorem{lemma}{Lemma}[section]
\newtheorem{corollary}[theorem]{Corollary}
\newtheorem{example}{Example}[section]
\newtheorem{definition}[problem]{Definition}
\newcommand{\BEQA}{\begin{eqnarray}}
\newcommand{\EEQA}{\end{eqnarray}}
\newcommand{\define}{\stackrel{\triangle}{=}}
\theoremstyle{remark}
\newtheorem{rem}{Remark}

% Marks the beginning of the document
\begin{document}
\bibliographystyle{IEEEtran}
\vspace{3cm}

\title{FUNCTIONS}
\author{EE24BTECH11038 - MALAKALA BALA SUBRAHMANYA ARAVIND}
\maketitle
\newpage
\bigskip

\renewcommand{\thefigure}{\theenumi}
\renewcommand{\thetable}{\theenumi}
\section{section b}
\begin{enumerate}
\item[19.]If the fractional part of the number $\frac{2^{403}}{15}$ is $\frac{k}{15}$, then k is equal to:

  \hfill {( JEE M 2019-9 Jan(M))}

   (A) 6   \hspace{2cm} (B)8\\

   (C)4  \hspace{2cm}  (D)14\\

\item[20.] If the function f:R-{-1,1}A defined by f(x)=$\frac{x^2}{1-x^2}$,is surjective then A is equal to:

   \hfill {( JEE M 2019-9 Jan(M))}

 (A)R-\{1\}  \hspace{2cm}  (B) (0,$\infty$)\\

 (B)R-[-1,0) \hspace{2cm}  (D)R-(-1,0)\\

    \item[21.] let $\sum\limits_{k=1}^{10}$f(a+k)=16($2^{10}$-1),where the function f satisfies f(x+y)=f(x)f(y) for all natural numbers x,y and f(a)=2.then the natural number 'a' is:

    \hfill {( JEE M 2019-9 April(M))}

    (A)2\hspace{1cm}(B)16\hspace{1cm}(C)4\hspace{1cm}(D)3
    
     
     
  \end{enumerate}
\end{document}
